\documentclass[12pt]{paper}
\usepackage{enumerate}

\title{A project proposal for an on-line interactive source code editing and evaluation tool.}
\author{Jennifer Wong, Marc Burns}

\begin{document}
	\maketitle
	\section{Introduction}
	There is currently debate over what toolchain and development environment shall be supported
	for use in first-year introductory computer science courses, specifically CS 136. In this proposal,
	the authors introduce an alternative solution which satisfies the use case in CS 136, and remains cross-platform
	with low installation and support overhead.

	The authors intend to create an application which allows for editing, compilation, testing, and submission of
	assignment solutions from any modern\footnote{A browser capable of executing JavaScript code including the manipulation
  of a Document Object Model: Firefox, Google Chrome, Internet Explorer, Safari, Opera, and most other popular browsers
  satisfy this constraint.} web browser on any platform. This application will integrate with
	user storage in the undergraduate computing environment, and will allow direct submission to Marmoset.
	
	\section{Rationale}
	A major point of contention in the current choice of development environments is the complexity of configuration
	required for the tools used to run on three major platforms. Although the undergraduate environment and the testing
	harness used for student solutions both run Linux with native GNU tools, many students attempt to complete assignments
	on Windows or Mac OS X machines using the native tools of the respective platforms. In past terms, this has resulted
	in a challenging situation for course support and technical staff; the different toolchains impose nontrivial semantic
	differences on students' source code, and the process for successfully compiling and executing a simple C program
	is diverse across the different platforms.

	Previously, a virtual machine running Linux (Ubuntu 11.x) has been distributed to students. Although this resolved
	issues arising from diverse toolchains, there were stability and performance issues with the virtualization software
	used.

	In Fall 2012, a new compiler will be used which presents a homogeneous environment across three major platforms.
	However, there are still nontrivial installation and configuration issues on Windows machines.

	The project proposed here will offload compilation and testing to a standardized, remote environment. Variation
	between platforms will be localized to the web browser, which provides a fairly homogeneous and time-tested interface
	for cross-platform development. The route to project submission will be made shorter, not requiring users to log
  into Marmoset to begin testing.

	\section{Feasability}
  We believe the project, as described, is feasable for completion in approximately four months. In order to
  complete the user interface of the environment described, we would be using several open-source components\footnote{\begin{enumerate}[{\indent i.}]
              \setlength{\itemsep}{0pt}
              \setlength{\parskip}{0pt}
              \item CodeMirror, a rich in-browser source editor ({\tt http://codemirror.net/})
              \item jQuery, a popular user-interface library for in-browser JavaScript applications\\({\tt http://jquery.com/})
              \item TermLib, a JavaScript object providing in-browser terminal emulation\\({\tt http://www.masswerk.at/termlib})
            \end{enumerate}}.

  A similar proof-of-concept was completed by Marc Burns and Jacob Parker in Winter 2012: The authors here\footnote{A working example of the in-browser MIPS assembler and VM:\\
            \indent\indent (\tt{http://csclub.uwaterloo.ca/$\sim$m4burns/evalc/mips/index.html})}
  created a simple MIPS VM, source editor, and assembler running completely in the browser.
	
\newpage
	\section{Potential Issues}
  We have yet to investigate the security and performance implications of this project. It will be necessary to
  interface with various infrastructure components, including Marmoset, Kerberos, LDAP, and CAS. Any release
  of the software usable by students must ensure privacy, data integrity, and moderate to good performance under
  load.

	\section{Plan}
  We first plan to investigate the security and infrastructure implications of the system described here.
  If the design, as described here, is found to be feasable for implementation, we will sketch our implementation
  and begin preparing a beta candidate. Otherwise, we will revise the design. Testing and review will occur once
  a candidate implementation has been finished, and then development will continue. Unit tests will be used
  throughout development.

  Given the components we may choose to use, this project would be compatible with an MIT license.
	
	\section{Limitations}
  In order to make use of this project, students would require access to a JavaScript-enabled web browser and network connection.

\end{document}
